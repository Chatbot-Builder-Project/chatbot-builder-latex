\section{Infrastructure}
\label{sec:infrastructure}

\subsection{Architectural Approach}
The infrastructure was designed using Infrastructure-as-Code (IaC) principles with three core components:

\begin{itemize}
    \item \textbf{Terraform}: For provisioning cloud resources on Azure, leveraging Terraform Cloud for state management.
    \item \textbf{Kubernetes}: For container orchestration and deployment management using Azure Kubernetes Service (AKS).
    \item \textbf{GitHub Actions}: For CI/CD pipeline automation, managing infrastructure provisioning and application deployment.
\end{itemize}

This combination enables reproducible environments and automated deployment processes across staging and production environments.

\subsection{Terraform Implementation}
The Terraform configuration manages the complete Azure infrastructure through modular components:

\begin{itemize}
    \item \textbf{Resource Groups}: Logical containers for Azure resources.
    \item \textbf{AKS Clusters}: Kubernetes clusters with auto-scaling node pools.
    \item \textbf{Blob Storage}: Azure Blob Storage for persistent data storage.
    \item \textbf{Public IPs}: Dedicated IP addresses per environment, injected into Kubernetes manifests.
\end{itemize}

\begin{lstlisting}[language=HCL,caption=Sample Terraform module for AKS]
module "aks" {
  source              = "./modules/aks"
  environment         = "staging"
  node_count         = 3
  vm_size            = "Standard_D2_v2"
  dns_prefix         = "chatbot-builder-staging"
  resource_group_name = azurerm_resource_group.main.name
}
\end{lstlisting}

Terraform outputs include:
\begin{itemize}
    \item \textbf{Kubernetes Configuration}: Kubeconfig file for cluster authentication.
    \item \textbf{Public IPs}: Used for Load Balancers and injected into Kubernetes manifests.
\end{itemize}

\subsection{Kubernetes Orchestration}
The Kubernetes implementation follows a multi-environment strategy using Kustomize:

\begin{itemize}
    \item \textbf{Base Configuration}: Common resources (Pods, Services, Ingress).
    \item \textbf{Environment Overlays}: Staging and Production-specific customizations.
    \item \textbf{Secret Management}: Kubernetes secrets encrypted with SealedSecrets.
\end{itemize}

Deployment follows this phased approach:
\begin{enumerate}
    \item Persistent Volume Claims for storage.
    \item ConfigMaps for environment variables.
    \item Microservice deployments.
    \item Network policies and ingress rules.
\end{enumerate}

Directory structure:
\begin{itemize}
    \item \textbf{base/}: Core manifests including ConfigMaps, PVCs, Deployments, and Services.
    \item \textbf{overlays/}: Staging and production configurations.
    \item \textbf{secrets/}: Templates for Kubernetes secrets, applied manually before automation.
\end{itemize}

\subsection{CI/CD Pipeline}
The CI/CD pipeline is managed with GitHub Actions and follows GitOps principles.
Key workflows:

\begin{itemize}
    \item \textbf{Terraform Deployment}: Runs on changes to the `infra/` directory.
    \begin{itemize}
        \item Initializes Terraform and validates configuration.
        \item Applies infrastructure changes if on `main` branch.
        \item Outputs kubeconfig and public IPs.
    \end{itemize}
    \item \textbf{Kubernetes Deployment}: Triggered via repository dispatch for staging/production.
    \begin{itemize}
        \item Fetches Terraform outputs (Kubernetes config, public IPs).
        \item Updates Kubernetes manifests with correct environment values.
        \item Deploys microservices and infrastructure resources to AKS.
    \end{itemize}
\end{itemize}
