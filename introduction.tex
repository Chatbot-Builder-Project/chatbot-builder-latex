\chapter{Introduction}

\section{Background}
Chatbots have become a cornerstone of modern digital interactions, bridging the gap between businesses and users through automated, yet personalized communication. Initially designed as rule-based systems with rigid decision trees, chatbots have evolved into dynamic agents powered by artificial intelligence (AI) and natural language processing (NLP). Despite this evolution, the current landscape of chatbot development tools remains fragmented. Solutions today are polarized: they either focus on static, flow-based designs (e.g. ManyChat, Chatfuel) or fully AI-driven interactions (e.g., Dialogflow, Rasa), with few attempts to integrate both approaches effectively. This divide limits developers' ability to create chatbots that are both predictable in structured workflows and adaptable to dynamic user input. Furthermore, existing tools often lack critical features such as advanced customization, extensibility through marketplaces, and flexible deployment options—gaps that hinder innovation and scalability in chatbot development.

\section{Terminology}
To ensure clarity throughout this report, key terms and concepts are defined below:

\subsection{Core Concepts}
\begin{itemize}
    \item \textbf{Chatbot}: A software application designed to simulate human conversation through text or voice interactions, often used for customer support, information retrieval, or task automation.
    
    \item \textbf{Static Workflow}: A predefined conversation path using decision trees and fixed responses, offering predictable and structured interactions.
    
    \item \textbf{Dynamic Routing}: The ability to determine conversation flow based on AI analysis of user intent rather than predefined rules.
    
    \item \textbf{Hybrid Approach}: The combination of static workflows with dynamic, AI-powered interactions in a single chatbot system.

\end{itemize}

\subsection{Technology Stack}
\begin{itemize}
    \item \textbf{LLM (Large Language Model)}: Advanced AI models trained on vast datasets to generate human-like text and understand context.
    
    \item \textbf{LangChain}: A framework for developing applications powered by language models, used in our executor service.
    
    \item \textbf{Microservices}: An architectural style where the application is structured as a collection of loosely coupled services.
    
    \item \textbf{gRPC}: A high-performance RPC (Remote Procedure Call) framework used for communication between services.
    
    \item \textbf{Infrastructure as Code (IaC)}: The practice of managing and provisioning infrastructure through code rather than manual processes.
\end{itemize}

\subsection{Development Concepts}
\begin{itemize}
    \item \textbf{No-Code/Low-Code}: Development approach that enables users to create applications through visual interfaces rather than traditional programming.
    
    \item \textbf{Visual Builder}: A drag-and-drop interface for creating and modifying chatbot workflows without coding.
    
    \item \textbf{Visual Editor}: Interface for customizing the chatbot's appearance and user interface elements.
\end{itemize}

\section{Objectives}
The primary objectives of this work are:
\begin{itemize}
    \item \textbf{To analyze existing chatbot development tools} and identify gaps in their ability to combine static workflows with dynamic adaptability, customization, extensibility and deployment options.
    
    \item \textbf{To design and develop FlowX}, a visual building tool that integrates static, rule-based workflows with dynamic, LLM-powered interactions, addressing the identified gaps.
    
    \item \textbf{To provide advanced customization tools} for chatbot design, allowing users to customize layouts, colors, fonts, and interactive elements to align with brand identity and user experience goals.
    
    \item \textbf{To create a marketplace ecosystem} for sharing, discovering and extending chatbot functionalities, fostering collaboration and reuse.
    
    \item \textbf{To simplify deployment} through flexible publishing options, including API integration and marketplace listings, ensuring accessibility across platforms.
\end{itemize}

\section{Summary}
FlowX addresses critical gaps in the chatbot development landscape by offering a unified platform that combines structured workflows with dynamic adaptability. Its emphasis on customization, extensibility, and ease of use makes it accessible to both technical and non-technical users, democratizing the creation of intelligent chatbots. By bridging the divide between static and AI-driven approaches, FlowX has the potential to improve scalability, reduce development costs, and improve user engagement across industries such as e-commerce, healthcare, and customer service.
